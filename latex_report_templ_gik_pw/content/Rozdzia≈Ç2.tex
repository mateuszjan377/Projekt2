\section{Przebieg projektu}
\subsection{GitHub} 
Na samym początku zaczeliśmy od stworzenia publicznego repozytorium, po czym utorzyliśmy wtyczkę której plik zaczeliśmy śledzić na repozytorium. Utorzyliśmy nowy token, dzięki któremu mogliśmy wspólnie hostować pliki na github. Po skończeniu projektu utorzyliśmy readme.md na githubie w naszym repozytorium.
\subsection{Przewyższenie między dwoma punkatmi} 
Do wykonania tych obliczeń niezbędne było stworzenie własnej warstwy, na której znajdują się punkty o znanych współrzędnych i wysokościach. Konieczne było również zrobienie przycisku w QT designer, który steruje daną funkcją.
\subsection{Pole powierzchni między conajmniej 3 punkatmi}   
Tak jak w przypadku przewyższenia należało wykorzystać utworzoną wcześniej warstwę a na niej conajmniej 3 punkty o znanych współrzędnych, z  ktorych zostalo obliczone pole powierzchni.
\subsection{Zliczenie elementów}
Dodaliśmy od siebie dodatkową funkcje zliczającą liczbe punktów na mapie 
