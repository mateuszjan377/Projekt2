\documentclass[10pt,a4paper]{article}

% ------------------------- PREAMBUŁA -------------------
\input{settings/pakiety}   % ścieżka względna do katalogu z ustawieniami
\graphicspath{{images/}}    % stała ścieżka względna do katalogu z  obrazkami.



% METADATA
% DOCUMENT METADATA
\newcommand{\logoGIK}{settings/WGiK-znak.png}
\newcommand{\authorName}{Mateusz Jankowski 319323, Kacper Kędra 319331  \\ grupa 2a}

\newcommand{\titeReport}{Projekt transformacje} % <<< here insert short title in the food
\newcommand{\titleLecture}{Informatyka Geodezyjna \\ sem. IV, ćwiczenia, rok akad. 2022-2023} % <<< insert title of presentation
\newcommand{\kind}{report}
\newcommand{\mymail}{\href{mailto:01169857@pw.edu.pl,01169867@pw.edu.pl}{01169857@pw.edu.pl, 01169867@pw.edu.pl}}
\newcommand{\supervisor}{....}
\newcommand{\gikweb}{\href{www.gik.pw.edu.pl}{www.gik.pw.edu.pl}}
\newcommand{\institut}{Zakład Geodezji Wyższej i Astronomii}
\newcommand{\faculty}{Wydział Geodezji i Kartografii}
\newcommand{\university}{Politechnika Warszawska}
\newcommand{\city}{Warszawa}
\newcommand{\thisyear}{2022}
%\date{}
% PDF METADATA
\pdfinfo
{
  /Title       (GIK PW)
  /Creator     (TeX)
  /Author      (Imię Nazwisko)
}


% ------------------------- POCZATEK DOKUMENTU -------------------
\begin{document}
% ----------------------------------------------------------------
% ----------------------------  Title page
% ----------------------------------------------------------------
\begin{center} 
\rule{\textwidth}{.5pt} \\
\vspace{1.0cm}
    \includegraphics[width=.4\paperwidth]{\logoGIK}
\vspace{0.5cm} \\
	\Large \textsc{\titeReport}
\vspace{0.5cm} \\  
	\large \textsc{\titleLecture}
\vspace{0.5cm}\\
	\textsc{\authorName}  \\
	\mymail \\
	\textsc{\faculty}, \textsc{\university}  \\ 
	 \city, \today
\end{center} 
\rule{\textwidth}{1.5pt}



% ---------------------------------------------------------------
% ----------------------------  Table of content
% ----------------------------------------------------------------
\tableofcontents 								% wyświetla spis treści
%\addcontentsline{to}{chapter}{Spis treści} 	% dodaje pozycję do spisu treści
% \listoffigures  								% wyświetla spis rysunków
%\addcontentsline{toc}{chapter}{Lista rysunków} % dodaje pozycję do spisu treści
% \listoftables 									% wyświetla spis rysunków
%\addcontentsline{toc}{chapter}{lista tabel}	% dodaje pozycję do spisu treści
\newpage
%


\section{Opis ćwiczenia}
\subsection{Cel projektu}
Celem projekt było stworzenie w programie QGIS wtyczki, która pozwala na obliczanie przewyższeń między punktami oraz obliczenie pola powierzchni między conajmniej 3 punktami
\subsection{Wykorzystane narzędzia i materiały}
Aby utorzyć wtyczkę w QGIS potrzebne nam były dwie wtyczki pomocnicze: plugin builder i plugin reloader, które musieliśmy pobrać w programie QGIS. Niezbędne było również wykorzystanie programu PYTHON oraz QT designer
\section{Przebieg projektu}
\subsection{GitHub} 
Na samym początku zaczeliśmy od stworzenia publicznego repozytorium, po czym utorzyliśmy wtyczkę której plik zaczeliśmy śledzić na repozytorium. Utorzyliśmy nowy token, dzięki któremu mogliśmy wspólnie hostować pliki na github. Po skończeniu projektu utorzyliśmy readme.md na githubie w naszym repozytorium.
\subsection{Przewyższenie między dwoma punkatmi} 
Do wykonania tych obliczeń niezbędne było stworzenie własnej warstwy, na której znajdują się punkty o znanych współrzędnych i wysokościach. Konieczne było również zrobienie przycisku w QT designer, który steruje daną funkcją.
\subsection{Pole powierzchni między conajmniej 3 punkatmi}   
Tak jak w przypadku przewyższenia należało wykorzystać utworzoną wcześniej warstwę a na niej conajmniej 3 punkty o znanych współrzędnych, z  ktorych zostalo obliczone pole powierzchni.
\subsection{Zliczenie elementów}
Dodaliśmy od siebie dodatkową funkcje zliczającą liczbe punktów na mapie 

\section{Podsumowanie}


\subsection{Umiejętności nabyte w trakcie ćwiczenia}
- Tworzenie wtyczki w programie QGIS
\\-	umiejętność robienia własnej wtyczki w python
\\- tworzenie wtyczki w sposób graficzny w programie QT designer 
\\- polepszenie współpracy na github

\subsection{Spostrzeżenia i trudności}
- Głownym napotkanym problemem było utworzenie własnej wtyczki koniecznej do obliczenia pola powierzchni oraz przewyższeń 
\\- Kolejnym problemem było obliczenie przewyższeń i pola powierzchni w programie QGIS 
\subsection{link do repozytorium}
https://github.com/mateuszjan377/Projekt2.git





% ----------------------------------------------------------------
% ----------------------------  Bibliography  
% ----------------------------------------------------------------

\end{document}
